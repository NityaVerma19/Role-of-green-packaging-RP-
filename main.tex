\documentclass[a4paper,10pt]{article}

\usepackage{fancyhdr} % clears the header and footer setting
\usepackage{graphicx} 
\usepackage{geometry}
\geometry{a4paper, left=2cm, right=2cm, top=1.5cm, bottom=3cm }
\usepackage{caption}
\usepackage{subcaption}
\usepackage{hyperref}
\usepackage{natbib}

\usepackage{etoolbox,fancyhdr,xcolor}
\newcommand{\headrulecolor}[1]{\patchcmd{\headrule}{\hrule}{\color{#1}\hrule}{}{}}
\newcommand{\footrulecolor}[1]{\patchcmd{\footrule}{\hrule}{\color{#1}\hrule}{}{}}
\renewcommand{\headrulewidth}{1pt}
\headrulecolor{red!100}%
\renewcommand{\footrulewidth}{1pt}
\footrulecolor{red!100}%

\fancyhf{}
\fancyhead[R]{\includegraphics[width=0.25\textwidth]{nmims.png}}

\fancyfoot[C]{Nilkamal School of Mathematics, Applied Statistics & Analytics}
\fancyfoot[R]{\thepage}

\setlength{\headheight}{15mm}
\pagestyle{fancy} %This command sets the page style to fancy, applying the header and footer configurations defined above to the document pages.

\bibliographystyle{apacite}

\usepackage{times}
\begin{document}

\noindent 
\begin{center}
\textbf{{\Large Role of Green Packaging In Consumer Purchasing Decisions using Logistic Regression}} \\
\end{center}

\noindent 
\textbf{Nitya Verma,} \textit{Narsee Monjee Institute Of Management Studies, Mumbai}\\

\noindent 
\textbf{ABSTRACT: } The need for sustainable packaging is growing rapidly. Non-biodegradable or traditional packaging materials leave carbon footprints which lead to degradation of the environment. This paper aims to find the whether the consumers are inclined towards buying products with sustainable packaging or not. The study uses logistic regression to classify and predict the consumers decisions. It was found that younger customers and customers belonging to high income class were generally the ones who purchased products with green packaging. 
 

\noindent 
\textbf{KEYWORDS:} \textit{Green  packaging},\textit{Logistic Regression}

\section{INTRODUCTION}

Consumers are the essence of any business. They come at the top of the development process of every business \cite{Witek2020}. The phrase customer is king is valid in every venture across the globe as they make or break a business. The importance given to the customers makes it all the more crucial to understand their psyche and behaviour as that directly impacts the firms' profit and longevity. 

It's no surprise that environmental contamination is at its peak currently \cite{RaviVyas2023}. An enormous bulk of this contamination comes from waste packaging. Due to the growing age of the internet, the social and environmental awareness of consumers is growing rapidly \cite{Groening2018}. People care more about the state of the world now than ever before. 

Corporate Social Responsibility (CSR) activities were made mandatory in April 2014 under the Companies Act 2013. The main aim of this modification by the Government of India was to inculcate a sense of social consciousness in the firms. The increase in environmental degradation made it all the more vital for the firms to make small changes to influence society at large, which led to the popularity of a new term known as "Green Marketing". Green Marketing came into emergence in the 1980s. It simply means the promotion, distribution, and packaging of products and services by focusing on their sustainability and ecological benefits. 

Green Marketing comes with its own set of pros and cons but the perspective matters. Looking from a broader outlook, Green Marketing has more pros than drawbacks. The positive effect it has on the environment in the long run, a better brand image, and appeal to environmentally conscious customers. It becomes crucial for organizations and businesses to pay more attention to green marketing as it plays a huge role from a competitive perspective as well. Consumers would naturally opt for environment-friendly products \cite{Groening2018}


\subsection{Green Packaging}

Green packaging is a subset of green marketing. It refers to the usage of eco-friendly packaging instead of plastics and non-biodegradable items. Ministry of Food Processing Industries published a focus paper, \cite{MOFPI2023} which gave insights on sustainable packaging. Some of the major benefits of green packaging mentioned in the paper were minimizing carbon footprint, waste reduction, and improvement in brand image. Food, normally requires plenty of packaging which in turn requires excessive materials. Paper packaging, biodegradable plastics, and compostable plastics are some of the common green packaging products. \cite{Oloyede2021} says that paper is one of the best materials for packaging as it has a significant value and is better in terms of 
eco-friendliness as compared to rest of the materials.

\section{Methodology}

\subsection{Logistic Regression}

Logistic regression is a supervised learning technique used for classification problems. It helps in classifying data into classes on the basis of independent variables. The dependent (target) variable is binary in nature that is it has two outcomes. 
Logistic regression makes the use of sigmoid function which helps in the transformation of any real-valued number into a number between 0 and 1. Figure \ref{sigmoid} shows an S-shaped graph of a sigmoid function. 


\begin{figure}[ht]
\centering
\includegraphics[height=6.6cm]{figures/sigmoid.png}
\caption{Sigmoid Function}
\label{sigmoid}
\end{figure}

(\ref{sigmoid_eq}) shows the equation of a sigmoid function where $x$ is any real-valued number which needs to be converted into a number between 0 and 1

\begin{equation}
f(x) = \frac{1}{1 + e^{-x}}
\label{sigmoid_eq}
\end{equation}

\subsection{Data}
\subsubsection{Data Collection}

The data is stimulated using Python and AI tools. An overview of the variables considered in the data was taken by going through different research papers and articles. 

\subsubsection{Data Summary}
The sample encompasses the purchase decisions made by customers regarding various products, which may or may not feature green packaging.The dataset has 330 customers and their information. 
The data consists of the following variables
\begin{itemize}

    \item Income 
    \item Age 
    \item Gender 
    \item Price of the product
    \item Type of product
    \item Green packaging (Yes/No)
    \item Purchase Decision (Yes/No)
\end{itemize}

Whether the customer purchased the product or not is the dependent variable and the rest are independent variables. 

\begin{table}[h]
\centering
\caption{Summary Statistics}
\begin{tabular}{|l|l|l|l|l|l|l|}
\hline
\textbf{Index} & \textbf{Age} & \textbf{Income} & \textbf{Gender} & \textbf{Price} & \textbf{Green Pack} & \textbf{Purchase} \\
\hline
Count & 330 & 330 & 330 & 330 & 330 & 330 \\
\hline
Mean & 40.188 & 79093.939 & 0.500 & 2305.558 & 0.545 & 0.661 \\
\hline
Std & 10.561 & 31987.012 & 0.501 & 1251.297 & 0.499 & 0.474 \\
\hline
Min & 20 & 25000 & 0 & 300 & 0 & 0 \\
\hline
25\% & 31 & 52250 & 0 & 1200 & 0 & 0 \\
\hline
50\% & 40 & 78000 & 0.5 & 2050 & 1 & 1 \\
\hline
75\% & 49 & 98000 & 1 & 3206.75 & 1 & 1 \\
\hline
Max & 60 & 160000 & 1 & 4932 & 1 & 1 \\
\hline
\label{summary_statistic}
\end{tabular}
\end{table}

Table \ref{summary_statistic} shows the summary statistic of all the variables. The average price is  2305.558 and the mean income of the customers is 79093.939. The lowest income is 25000 and the maximum income is 160000



\begin{table}[h]
\centering
\caption{Correlation Matrix}
\begin{tabular}{|l|l|l|l|l|l|l|}
\hline
 & \textbf{Age} & \textbf{Income} & \textbf{Gender} & \textbf{Price} & \textbf{Green Pack} & \textbf{Purchase} \\
\hline
\textbf{Age} & 1.00 & 0.87 & -0.07 & 0.69 & -0.17 & 0.33 \\
\hline
\textbf{Income} & 0.87 & 1.00 & -0.03 & 0.81 & -0.14 & 0.41 \\
\hline
\textbf{Gender} & -0.07 & -0.03 & 1.00 & -0.04 & -0.16 & -0.06 \\
\hline
\textbf{Price} & 0.69 & 0.81 & -0.04 & 1.00 & -0.21 & 0.35 \\
\hline
\textbf{Green Pack} & -0.17 & -0.14 & -0.16 & -0.21 & 1.00 & -0.06 \\
\hline
\textbf{Purchase} & 0.33 & 0.41 & -0.06 & 0.35 & -0.06 & 1.00 \\
\hline
\label{Correlation_matrix}
\end{tabular}
\end{table}

Table \ref{Correlation_matrix} shows the correlation among the numerical variables. It is evident from the table that the \textit{Income} of the customers and the \textit{Purchase decision} of the customers are positively correlated. 
It is also apparent that there is a negative correlation between the \textit{age} of the customers and the decision to \textit{buy} the product



\subsection{Case Analysis}


\begin{figure}[ht]
\centering
\includegraphics[height=6.6cm]{figures/coeff.png}
\caption{Parameters of the logistic regression model and their assessment}
\label{fig_regression}
\label{summary_lr}
\end{figure}

\ref{summary_lr} shows the variables present in the data and their respective coefficients. It is evident that age and income are significant at 0.05 .We will now estimate the parameters of the logistic regression model, the values of which are given.

\begin{equation}
P(\text{Purchase} = \text{yes} | X) = \frac{e^{a}}{1 + e^{a}}
\label{LR_model}
\end{equation}
where
\begin{equation}
a = -0.0899 \times \text{age} + 6.592 \times 10^{-5} \times \text{income} - 0.4696 \times \text{gender} - 7.279 \times 10^{-5} \times \text{price} - 0.302 \times \text{green\_pack}
\label{LR_equation}
\end{equation}

\ref{class_tab} shows the classification table. It can be seen that 66.67\% of the data is correctly classified. Only age and income are significant in explaining the decision of the consumer to purchase the product. The negative sign of the age coefficient indicates that older people generally don't buy products with green packaging and the positive coefficient of income shows that people with higher income mostly purchase the products with green packaging.


\begin{figure}[ht]
\centering
\includegraphics[height=4.5cm]{figures/confusion_matrix.png}
\caption{Classification Table}
\label{class_tab}
\end{figure}
\vspace{10pt}
\ref{accuracy} shows the various accuracy measures to better understand the model


\begin{itemize}
    \item 	Precision: Precision measures the accuracy of positive predictions made by the model.
    \begin{equation}
        \text{Precision} = \frac{\text{True Positives}}{\text{True Positives} + \text{False Positives}}
    \end{equation}
    For class 0 (no purchase), it means that among the instances predicted as "no purchase" by the model, 60\% of them are actually "no purchase". Similarly, for class 1 (purchase), 69\% of instances predicted as "purchase" by the model are actually "purchase".
    \vspace{15pt}

    \item Recall/Sensitivity : Recall measures the ability of the model to correctly classify the positive instances.
       \begin{equation}
        \text{Sensitivity} = \frac{\text{True Positives}}{\text{True Positives} + \text{False Negatives}}
    \end{equation}
   The sensitivity of the model turns out to be 85\% which means that the model correctly identifies 85\% of the actual "purchase" instances.
    \vspace{15pt}

    \item Specificity : Specificity is an accuracy measure which represents the ability of the model to correctly classify the negatives instances. 
    \begin{equation}
        \text{Specificity} = \frac{\text{True Negatives}}{\text{True Negatives} + \text{False Positives}}

    \end{equation}
    
The specificity of the model is 36\% which means that for the "No purchases" class the model correctly identifies 36\% of the actual "no purchase" instances.
     \vspace{15pt}
 
    \item The F1-score is the harmonic mean of precision and recall. It provides a balance between precision and recall. For class 0 (no purchase), the F1-score is 0.45, and for class 1 (purchase), the F1-score is 0.76.

\end{itemize}

\begin{figure}[ht]
\centering
\includegraphics[height=4.5cm]{figures/accuracy.png}
\caption{Classification Report} 
\label{accuracy}
\end{figure}

\vspace{10pt}
\begin{figure}[ht]
\centering
\includegraphics[height=4.5cm]{figures/AUC_ROC.png}
\caption{Receiver Operating Characteristic Curve} 
\label{auc}
\end{figure} 

\vspace{30pt}
\ref{auc} shows the Receiver operating characteristic curve and the Area Under the curve of the model.
\begin{itemize}
    \item AUC is the proportion of the concordant pairs in the data if the model is used for classification. AUC for the data turned out to be \textit{0.64} which refers to \textbf{weak discrimination}. The model's performance is fair but there is room for improvement

    \item ROC curve can be used to understand the overall worth of the logistic regression model. Sensitivity is on the Y-axis and (1-Specificity) is on the X-axis. The straight line shows the 0.5 cut-off.
\end{itemize}


\section{Conclusion}
\begin{itemize}
    \item From the case it can be seen that the consumers belonging to an high income class are the ones who generally purchase products with green packaging. 
    \item Younger customers are more likely to purchase products with green packaging than older customers. 
    \item More awareness regarding green packaging should be spread in order to decrease and eventually eliminate non-recyclable waste.
\end{itemize}

\section{Future Scope}

As the data was stimulated for the purpose of understanding and applying logistic regression, the accuracy couldn't be high. More variables and real-life data can be taken into account to make more realistic and better prediction.



\fontsize{8}{9}\selectfont
\bibliography{ResearchPaperBib}



\clearpage



\end{document}